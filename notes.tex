\documentclass[12pt]{article}

\addtolength{\textwidth}{1.4in}
\addtolength{\oddsidemargin}{-.7in} %left margin
\addtolength{\evensidemargin}{-.7in}
\setlength{\textheight}{8.5in}
\setlength{\topmargin}{0.0in}
\setlength{\headsep}{0.0in}
\setlength{\headheight}{0.0in}
\setlength{\footskip}{.5in}
\renewcommand{\baselinestretch}{1.0}
\setlength{\parindent}{0pt}
\linespread{1.1}

\usepackage{amssymb, amsmath, amsthm, bm}
\usepackage[normalem]{ulem}
\usepackage{graphicx,csquotes,verbatim}
\usepackage[backend=biber,block=space,style=authoryear]{biblatex}
\setlength{\bibitemsep}{\baselineskip}
\usepackage[american]{babel}
%dell laptop
\addbibresource{C:/Users/Kristy/Dropbox/Research/xBibs/tradeagreements.bib}
%\addbibresource{C:/Users/Kristy/Documents/Dropbox/Research/xBibs/tradeagreements.bib}
\renewcommand{\newunitpunct}{,}
\renewbibmacro{in:}{}


\usepackage[pdftex,
bookmarks=true,
bookmarksnumbered=false,
pdfview=fitH,
bookmarksopen=true,hyperfootnotes=false]{hyperref}

\DeclareMathOperator*{\argmax}{arg\,max}
\usepackage{xcolor}
\hbadness=10000

\newcommand{\ve}{\varepsilon}
\newcommand{\ov}{\overline}
\newcommand{\un}{\underline}
\newcommand{\ta}{\theta}
\newcommand{\al}{\alpha}
\newcommand{\expect}{\mathbb{E}}
\newcommand{\ga}{\gamma}
\newcommand{\Ga}{\Gamma}
\newcommand{\de}{\delta}

\begin{document}
\begin{center}
Gradualism Notes
\end{center}


\begin{comment}
\section{Big Picture}
\begin{itemize}
	\item From repeated game paper: if $\ga(e) \: \uparrow$, need lower $\tau^a$; so if $\ga(e) \; \downarrow$, can get away with lower $\tau^a$ and still be self-enforced
	\item Caroline Freund: lobbying wastes productive resources; that's what I've built in here
	\item Need feedback mechanism
		\begin{itemize}
			\item Need feedback from shock to lead to future lower investment
			\item Investment by lobby shifts political support function
				\begin{itemize}
					\item Investment complements/substitutes for shock?
				\end{itemize}
		\end{itemize}
\end{itemize}
\end{comment}

\begin{comment}
\section{Modeling choices}

\begin{enumerate}
	\item Small country: commitment to lower tariff, but no TOT change, just internal price
	\item Want to focus on lobby's dynamic problem
	\item Uncertainty vs certainty
		\begin{itemize}
			\item With certainty,  you know you'll get a given amount of support, i.e. know the price of maintaining support, which is getting your people re-elected (perhaps combined with getting them to vote your way)
				\begin{itemize}
					\item Think of $\ta$ as shocks to who's in office, which changes the price of maintaining support; it would take more $e$ to get a given $\ga$, either because you need to work harder to get other people elected next time, or you have to pay more to convince people who are fundamentally less agreeable to your cause
				\end{itemize}
		\end{itemize}
	\item Cobb-Douglas with fixed-factor (decreasing returns to scale)
		\begin{itemize}
			\item If I do CRS, then to get a solution/equilibrium, there is a fixed amount of labor that must be used to get zero profits given the price ratio.
			\item All reactions to changes in price are not really profit maximizing but to get the zero profit condition.
			\item Maybe tradeoffs between tariffs and investment in productivity for a given price, but again, it's baked into zero-profit condition
		\end{itemize}
	\item Tariff vs. quota
\end{enumerate}

\vskip.2in
Within each period $t$, taking initial wealth as given
\begin{enumerate}
	%\item[1a.] Firm productivity realized: $A(m_{t-1} + \mu_{t-1})$
	\item[1.] Election occurs (reduced form based on $e_{t-1}$)
	\item[2.] Lobby/firm chooses $l_t$ and makes investments in technology $\mu_t$ and politics $e_t$
	\item[3.] Government chooses tariff ($\tau_t$)
	\item[4.] Production takes place, workers are paid (profits realized)
	\item[5.] Tariff revenue is distributed and consumption takes place (not explicitly modeled)
\end{enumerate}
\end{comment}

\begin{comment}
\section{One-period model}
Given $\ga_0$
\[
  \max_{l_1,e_1,\mu_1} \ A(m_0 +\mu_1) \cdot F^\alpha \cdot l_1^{1-\alpha}\left[P^W + \tau\left(\gamma_0\right)\right] - l_1 - \mu_1 - e_1 
	\]


\vskip.2in
Interior F.O.C.'s
\[
   \left(1-\alpha\right) A(m_0 +\mu_1) \cdot \left(\frac{F}{l_1}\right)^\alpha \left[P^W + \tau\left(\gamma_0\right)\right] = 1
\]
\[
  \frac{\partial A(m_0 +\mu_1)}{\partial \mu_1} \cdot F^\alpha \cdot l_1^{1-\alpha}\left[P^W + \tau\left(\gamma_0\right)\right] = 1
\]
\[
  A(m_0 +\mu_1) \cdot F^\alpha \cdot l_1^{1-\alpha} \frac{\partial \tau}{\partial \ga} \frac{\partial \ga}{\partial e_1}= 1
\]
Combining these
\[
  \left(1-\alpha\right) A(m_0 +\mu_1) \cdot \left(\frac{F}{l_1}\right)^\alpha = \frac{\partial A(m_0 +\mu_1)}{\partial \mu_1} \cdot F^\alpha \cdot l_1^{1-\alpha}
\]
\[
  \left(1-\alpha\right) A(m_0 +\mu_1) = \frac{\partial A(m_0 +\mu_1)}{\partial \mu_1} \cdot l_1
\]
And
\[
  \left(1-\alpha\right) A(m_0 +\mu_1) \cdot \left(\frac{F}{l_1}\right)^\alpha \left[P^W + \tau\left(\gamma_0\right)\right] = A(m_0 +\mu_1) \cdot F^\alpha \cdot l_1^{1-\alpha} \frac{\partial \tau}{\partial \ga} \frac{\partial \ga}{\partial e_1}
\]
\[
  \left(1-\alpha\right) \left[P^W + \tau\left(\gamma_0\right)\right] = l_1 \frac{\partial \tau}{\partial \ga} \frac{\partial \ga}{\partial e_1}
\]
And combining these two
\[
  \frac{A(m_0 +\mu_1)}{P^W + \tau\left(\gamma_0\right)} = \frac{\frac{\partial A(m_0 +\mu_1)}{\partial \mu_1}}{\frac{\partial \tau}{\partial \ga} \frac{\partial \ga}{\partial e_1}}
\]
Notice there is no constraint assumed on $e_1$ or $\mu_1$.
\end{comment}


\vskip.5in
\section{Two-period model}
Given $\ga_0$
\begin{multline*}
  \max_{l_1,e_1,\mu_1,l_2,\mu_2} \ \left\{ A(m_0 +\mu_1) \cdot F^\alpha \cdot l_1^{1-\alpha}\left[P^W + \tau\left(\gamma_0\right)\right] - l_1 - \mu_1 - e_1 \right\} +\\
	\left\{ A(m_1 + \mu_2) \cdot F^\alpha \cdot l_2^{1-\alpha}\left[P^W + \tau\left(\gamma(e_1)\right)\right] - l_2 - \mu_2 \right\} 
\end{multline*}
where $m_1 = m_0 + \mu_1$

\vskip.3in
\subsection{Interior F.O.C.'s}
\begin{equation}
   l_1: \left(1-\alpha\right) A(m_0 +\mu_1) \cdot \left(\frac{F}{l_1}\right)^\alpha \left[P^W + \tau\left(\gamma_0\right)\right] = 1
	\label{eq:l1}
\end{equation}
\begin{equation}
  \mu_1: \frac{\partial A(m_0 +\mu_1)}{\partial \mu_1} \cdot F^\alpha \cdot l_1^{1-\alpha}\left[P^W + \tau\left(\gamma_0\right)\right] + \frac{\partial A(m_0 +\mu_1 + \mu_2)}{\partial \mu_1} \cdot F^\alpha \cdot l_2^{1-\alpha}\left[P^W + \tau\left(\gamma(e_1)\right)\right] = 1
	\label{eq:mu1}
\end{equation}
\begin{equation}
  e_1: A(m_0 +\mu_1 + \mu_2) \cdot F^\alpha \cdot l_1^{1-\alpha} \frac{\partial \tau}{\partial \ga} \frac{\partial \ga}{\partial e_1}= 1
	\label{eq:e1}
\end{equation}
\begin{equation}
  l_2: \left(1-\alpha\right) A(m_0 +\mu_1 + \mu_2) \cdot \left(\frac{F}{l_2}\right)^\alpha \left[P^W + \tau\left(\ga(e_1)\right)\right] = 1
	\label{eq:l2}
\end{equation}
\begin{equation}
  \mu_2: \frac{\partial A(m_0 +\mu_1 + \mu_2)}{\partial \mu_2} \cdot F^\alpha \cdot l_2^{1-\alpha}\left[P^W + \tau\left(\gamma(e_1)\right)\right] = 1
	\label{eq:mu2}
\end{equation}


\vskip.3in
\subsection{What happens when $\ga_0$ decreases?}
From Equation~\ref{eq:l1}, when $\ga_0 \! \downarrow$, $\frac{A(m_0 +\mu_1)}{l_1^\alpha} \! \uparrow$. Two cases:
	\begin{itemize}
		\item $\mu_1 \! \uparrow, \ l_1$ unchanged / $\mu_1 \! \uparrow \uparrow, \ l_1 \! \uparrow$ 
		\item $\mu_1 \text{ unchanged}, \ l_1 \! \downarrow$ / $\mu_1 \! \downarrow, \ l_1 \! \downarrow \downarrow$ 
	\end{itemize}


\vskip.3in
Two cases:
\begin{enumerate}
	\item $\mu_1 \! \uparrow$ and $l_1 \! \uparrow$: increase investment in productivity
		\begin{itemize}
			\item investment in politics ($e_1$) $\downarrow$
			\item $l_2 \! \uparrow$
		\end{itemize}
	\item $\mu_1 \! \downarrow$ and $l_1 \! \downarrow$: reduce investment in productivity
	  \begin{itemize}
			\item investment in politics ($e_1$) $\uparrow$
			\item $l_2 \! \downarrow$
		\end{itemize}
\end{enumerate}




\vskip.5in
\section{Literature}
\begin{itemize}
	\item Hillman (1991): decide between lobbying and investing in internal monitoring of production. In book at library, HF1372.158 1991 3rd Floor (Helpman and Razin)
	\item Rodrik (1996): use labor to make lobbying (p. 5/15)
	\item Krueger (1974)
	\item Sturzenegger F. (1993) never got published
\end{itemize}

\vskip.5in
\section{Feedback}
SEAs, Tampa, November 2017
\begin{itemize}
	\item Maia L. (Richmond): Why would lobby invest in technology?
	\item Woan Foon: what happens when there are more countries (as in GATT)?
	\item Kerem: why not $K$ instead of $A$?
	\item Kamal: changing preferences...
\end{itemize}


\end{document}